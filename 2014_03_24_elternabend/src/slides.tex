\documentclass[12pt]{beamer}
%\documentclass[20pt,handout]{beamer}
\usetheme{Darmstadt}
\usepackage{graphicx}
%\usepackage[german]{babel}
\usepackage[T1]{fontenc}
\usepackage[utf8]{inputenc}
\usepackage{tikz}
\setbeamertemplate{footline}[frame number]

\newcommand{\cc}[1]{\includegraphics[height=4mm]{img/#1.png}}
\usepackage{ifthen}
\newcommand{\license}[2][]{\\#2\ifthenelse{\equal{#1}{}}{}{\\\scriptsize\url{#1}}}
\usepackage{textcomp}

\pgfdeclareimage[height=.6cm]{c3d2logo}{./img/c3d2.pdf} 


\pgfdeclarelayer{foreground}
\pgfsetlayers{main,foreground}
\logo{\pgfputat{\pgfxy(-1,0)}{\pgfbox[center,base]{\pgfuseimage{c3d2logo}}}}


\title{Chancen und Gefahren des Internets}
\author{\small Bernhard Franke \& Marius Melzer\\\large Chaos Computer Club Dresden}
\date{24.03.2014}

\begin{document}
\maketitle

\section{Einleitung}
\subsection{}

\begin{frame}
  \frametitle{Wer sind wir?}
  \begin{itemize}
    \item<1-> Chaos Computer Club Dresden (\url{http://c3d2.de})
        \note{}
    \item<2-> Datenspuren: 13./14.09. 2014 \url{http://datenspuren.de}
    \item<3-> Podcasts (\url{http://pentamedia.org})
    \item<4-> Chaos macht Schule
  \end{itemize}
\end{frame}

\section{Gefahren}
\subsection{}

\begin{frame}
  \frametitle{Gefahren}
  \begin{itemize}
    \item<2-> Kontrollverlust über eigene Daten
    \item<3-> Identitätsdiebstahl
    \item<4-> Unerwünschte Inhalte
    \item<5-> Cybermobbing
    \item<6-> Urheberrechtsverletzungen
  \end{itemize}
\end{frame}

\begin{frame}
  \frametitle{Kontrollverlust über eigene Daten}
  \begin{itemize}
    \item<2-> Alles, auch jeder Websiteaufruf, ist eine Kopie
    \item<3-> => Das Internet vergisst nichts 
    \item<4-> Beispiel 1: archive.org
    \item<5-> Beispiel 2: NSA \& co
    \item<6-> Nicht unwahrscheinlich, dass eigene Daten in Jahren oder Jahrzehnten noch abrufbar/vorhanden sind
  \end{itemize}
\end{frame}

\begin{frame}
  \frametitle{Geschäftsmodelle}
  \begin{itemize}
    \item<2-> Daten haben einen Wert
    \item<3-> Womit verdienen folgende Firmen ihr Geld?
      \begin{itemize}
        \item<4-> Karstadt
        \item<5-> Amazon
        \item<6-> Ebay
        \item<7-> Facebook
      \end{itemize}
      \item<8-> Werbung verkauft sich personalisiert besser -> Personalisieren durch Tracking
  \end{itemize}
\end{frame}

\begin{frame}
  \frametitle{Gegenmaßnahme: Datensparsamkeit}
  \begin{itemize}
      \item<2-> Viele Daten zusammen ergeben Profile
      \item<3-> Werden die Daten gebraucht?
      \item<4-> Werden echte Daten gebraucht?
          \begin{itemize}
            \item<5-> Pseudonymität
            \item<6-> mailinator.com (Wegwerf-Email-Adresse)
            \item<7-> frank-geht-ran.de (Wegwerf-Telefonnummer)
            \item<8-> bugmenot.com (Fake Accounts)
          \end{itemize}
  \end{itemize}
\end{frame}

\begin{frame}
    \frametitle{Identitätsdiebstahl: Passwörter}
    \begin{itemize}
        \item<2-> Keine einfachen Wörter
        \item<3-> Groß-, Kleinbuchstaben, Ziffern, Sonderzeichen
        \item<4-> Beispiele:
            \begin{itemize}
                \item<5-> dragon
                \item<6-> (nCuAj.§Tsm!f
                \item<7-> IchLiebeDich
                \item<8-> .§)=")=`
                \item<10-> qwerty
                \item<11-> Mks?o/.u,1Psw!
            \end{itemize}
        \item<12-> Verschiedene Passwörter nutzen!
        \item<13-> Passwort-Manager verwenden \\ (z.B. Keepass, Password Safe)
    \end{itemize}
\end{frame}

\begin{frame}
  \frametitle{Unerwünschte Inhalte}
  \begin{itemize}
    \item<2-> Filter sind einfach zu umgehen
    \item<3-> Keine technische Lösung für soziale Probleme
  \end{itemize}
\end{frame}

\begin{frame}
  \frametitle{Cybermobbing}
  \begin{itemize}
    \item<2-> Kein großer Unterschied zu "normalem Mobbing"
    \item<3-> Nicht ignorieren, Lehrer und Schule können ggf. helfen
    \item<4-> EU-Programm: Klicksafe
  \end{itemize}
\end{frame}

\begin{frame}
  \frametitle{Urheberrecht I}
  \begin{itemize}
    \item<2-> Streaming im Graubereich
    \item<3-> "Peer-to-Peer" (z.B. Bittorrent) interessante Technologie, aber Nutzer transparent und Problem des Weiterverteilens
  \end{itemize}
\end{frame}

\begin{frame}
  \frametitle{Urheberrecht II}
  \begin{itemize}
    \item<2-> Alle eigenen Werke in Deutschland urheberrechtlich geschützt
    \item<3-> Public Domain und Creative Commons Lizenzen
    \item<4-> https://search.creativecommons.org
  \end{itemize}
\end{frame}

\section{Chancen}
\subsection{}

\begin{frame}
  \frametitle{Chancen}
  \begin{itemize}
    \item<2-> Internetnutzer ist nicht nur konsument, sondern auch Produzent
    \item<3-> Gigantische Wissensquelle
    \item<4-> Kommunikation, auch über Länder- und Kontinentgrenzen hinweg
  \end{itemize}
\end{frame}

\begin{frame}
  \frametitle{Internet als Wissensquelle: Wikipedia}
  \begin{itemize}
    \item<2-> Gemeinschaftlich aggregiertes Weltwissen
    \item<3-> ...unter freien Lizenzen
    \item<4-> Man kann sich selbst einbringen
    \item<5-> Besser als der Brockhaus?
    \item<6-> Keine "Quelle", aber mit Quellen hinterlegt
  \end{itemize}
\end{frame}

\begin{frame}
  \frametitle{Internet als Wissensquelle: OpenStreetMap}
  \begin{itemize}
    \item<2-> Gemeinschaftlich aggregierte Weltkarte
    \item<3-> ...unter freier Lizenz
    \item<4-> Man kann sich selbst einbringen
    \item<5-> Detaillierter als Google Maps?
  \end{itemize}
\end{frame}

\begin{frame}
  \frametitle{Internet als Wissensquelle: Nachrichten}
  \begin{itemize}
    \item<2-> Nachrichten auf Abruf
    \item<3-> Beispiele: tageschau.de, Zeit Online, Spiegel Online
  \end{itemize}
\end{frame}

\begin{frame}
  \frametitle{Internet als Kommunikationsmedium}
  \begin{itemize}
    \item<2-> Soziale Netzwerke sind im Grunde etwas Gutes
  \end{itemize}
\end{frame}

\begin{frame}
    \frametitle{Diskussion}
    \begin{center} {\Large Diskussion}\\Bernhard Franke und Marius Melzer\\CMS Dresden: schule@c3d2.de \end{center}
    \begin{center}
      \href{https://creativecommons.org/licenses/by-sa/4.0/}{\cc{by-sa}} \\
      \href{}{\textcolor{blue}{Folien}} vom Chaos Computer Club Dresden
    \end{center}
\end{frame}

\end{document}
