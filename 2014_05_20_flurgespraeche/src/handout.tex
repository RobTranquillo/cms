\documentclass[a5paper]{scrartcl}

\usepackage{hyperref}

\usepackage[german]{babel}
\usepackage[T1]{fontenc}
\usepackage[utf8]{inputenc}

\usepackage[a5paper, left=20mm, right=20mm, top=20mm, bottom=20mm]{geometry}

\begin{document}

\thispagestyle{empty}

\section*{Wie schütze ich mich vor Überwachung?}

\textbf{Kontakt:} schule@c3d2.de\\
\textbf{Website:} http://c3d2.de\\
\\
\subsection*{Test}
\begin{itemize}
  \item Ein Stichpunkt
    \begin{itemize}
      \item Ein eingerückter Stichpunkt
    \end{itemize}
\end{itemize}

\section*{E-Mail Verschlüsselung mit GNU Privacy Guard}
% Evtl. guter Einführungstext in Crypto: http://gpg4win.de/doc/de/gpg4win-compendium_8.html 
\subsection*{Plugins für Mailprogramme}
\begin{itemize}
   \item Mozilla Thunderbird
      \begin{itemize}
         \item \texttt{http://www.youtube.com/watch?v=CaXfruQo8Ks}  
      \end{itemize}       
   \item Apple Mail 
      \begin{itemize}
         \item \texttt{http://www.youtube.com/watch?v=6EtAJc51pGs}  
      \end{itemize}
   \item Microsoft Outlook 
      \begin{itemize}
         \item \texttt{http://www.youtube.com/watch?v=AfK1msNWr6s} 
      \end{itemize}
   \item K9 Mail
      \begin{itemize}
         \item \texttt{http://www.youtube.com/watch?v=N-qFEHJCU60} 
      \end{itemize}
\end{itemize}

\subsection*{Verschlüsselung im Browser mittels Mailvelope}
\begin{itemize}
    \item Google Chrome
       \begin{itemize}
               \item \texttt{http://www.youtube.com/watch?v=xKDk3l6nRc4}
       \end{itemize}
    \item Installation unter Firefox
       \begin{itemize}
               \item \texttt{https://addons.mozilla.org/en-US/firefox/addon/mailvelope} 
       \end{itemize}
\end{itemize}
\end{document}
