\documentclass[12pt]{beamer}
%\documentclass[20pt,handout]{beamer}
\usetheme{Darmstadt}
\usepackage{graphicx}
%\usepackage[german]{babel}
\usepackage[T1]{fontenc}
\usepackage[utf8]{inputenc}
\usepackage{tikz}
\setbeamertemplate{footline}[frame number]

\newcommand{\cc}[1]{\includegraphics[height=4mm]{img/#1.png}}
\usepackage{ifthen}
\newcommand{\license}[2][]{\\#2\ifthenelse{\equal{#1}{}}{}{\\\scriptsize\url{#1}}}
\usepackage{textcomp}

\pgfdeclareimage[height=.6cm]{c3d2logo}{./img/c3d2.pdf} 


\pgfdeclarelayer{foreground}
\pgfsetlayers{main,foreground}
\logo{\pgfputat{\pgfxy(-1,0)}{\pgfbox[center,base]{\pgfuseimage{c3d2logo}}}}


\title{Pentabug - Einführung in HAL}
\author{\small Paul Schwanse \\\large Chaos Computer Club Dresden}
\date{12.03.2015}

\begin{document}
\maketitle

\section{Einführung in HAL}

\begin{frame}
    \frametitle{Motivation}
    \begin{columns}[T]
	    \column{.5\linewidth}
    \begin{itemize}
	    \item tast

    \end{itemize}
    \column{.5\linewidth}
    \begin{itemize}
	    \item test
    \end{itemize}
    \end{columns}

\end{frame}

\end{document}
